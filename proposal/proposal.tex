\documentclass[12pt]{article}
\usepackage[margin=0.75in]{geometry}
\usepackage{sectsty}
\sectionfont{\fontsize{14}{15}\selectfont}
\usepackage{graphicx}
\usepackage{amsmath}
\DeclareMathOperator{\sign}{sign}
\usepackage{amssymb}
\usepackage{float}
\usepackage{enumitem}

\title{CSIE 5452, Fall 2018 — Final Project Proposal}
\author{Ching-Yuan Bai B05502055 \\Pin-Yen Huang B05902057}
\date{October 23, 2018}

\begin{document}

\maketitle

\section{Members}
\begin{tabular}{ |c|c|c| }
  \hline
  Name & Student ID & Emails \\
  \hline
  \hline
  Pin-Yen Huang & B05902057 & b05902057@ntu.edu.tw \\
  \hline
  Ching-Yuan Bai & B05502055 & b05502055@ntu.edu.tw \\
  \hline
\end{tabular}
\section{Type}
\begin{tabular}{ |c|c|c|c| }
  \hline
  Type & Survey & Implementation & Research \\
  \hline
  \hline
  \begin{tabular}{@{}c@{}}Estimated \\ Percentage\end{tabular} & 20\%  & 40\%  & 40\% \\
  \hline
\end{tabular}
\section{Title}
Pedestrian-focused autonomous intersection management
\section{Problem Description}
Attempt to devise an effective and efficient framework to incorporate pedestrain into autonomous intersection management.
\section{Reason to select the problem}
According to survey, there are currently no existing autonomous intersection management framework that considers the priority of pedestrain.
However, the time for pedestrains to wait at a crossroad should not be neglected as it inflicts no less social cost compared with automobiles.
Thus, a method to prioritized the pedestrians crossing intersections  while minimizing the negative effect on traffic flow is an urgent demand.
\section{Rough Schedule}
\begin{itemize}
  \item Nov, 18: Complete survey with full knowledge of state-of-the-art autonomous intersection management methods.
    Implement rudimentary model and experiment with toy dataset.
  \item Dec, 18: Obtain real world data to experimant and revise the model accordingly.
  \item Jan, 19: Complete the remaining pieces of the framework. Prepare for report. 
\end{itemize}
\section{Expected Results}
We expect to propose a novel method to incorporate pedestrains into the autonomous intersection management framework. 
It should be able to be theoretically provable better than conventional methods (worse case analysis) and follow the same trend in simulation as well. 
We plan to eveluate the feasibility of implementing it in real life situations.
\section{Any Double Assignment}
None.
\section{Any Completed Work}
None.
\section{Optional Questions}
\begin{itemize}
    \item What are the references that you want to survey?\\
      Journal and paper related to autonomous intersection management on Google Scholar.     
    \item How do you plan to survey the references (detailed or quick)?\\
      We plan to first go through the abstract and introduction of each paper and journal then dive deep into the more relevant ones.
    \item Is there any existing public implementation?
    \item How to you get the input data?
    \item What language do you want to use?\\
      Python.
    \item What are the references that you want to consider or compare with?\\
      We will build our model on top of the legacy by K. Dresner and P. Stone et al. (2008) and compare with the original version as baseline performance evalutation.
      Upon further maturity of the new model, we can compare with more advanced models such as the multi-intersection optimized model proposed by M Hausknecht et al..
      Alternative benchmarking can be done by comparing with other groups participating in this course with similar project topic.
    \item What are the possible directions that you can get improvement over existing work?
      \begin{itemize}
        \item Incorporate pedestrians into autonomous intersection management
        \item Utilize pedestrains to improve the autonomous intersection managing experience
        \item Possibly incorporate bicycles or anything traveling with different speed than automobiles 
        \item Propose solution to multiple targets with largely varying speed sharing the same route
      \end{itemize}
\end{itemize}

\end{document}
